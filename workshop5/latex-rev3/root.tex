\documentclass[8pt,oneside]{extarticle}

%\usepackage{subfigure}
\usepackage{subcaption}
\usepackage{tabularx}
\usepackage{graphicx}
\usepackage{amsmath}
\usepackage{amsfonts}
\usepackage{hyperref}
\usepackage{adjustbox}
\usepackage{listings}
\usepackage{optidef}
\usepackage{cleveref}
\usepackage{threeparttable}
\usepackage{xcolor}
\usepackage{titlesec}
\usepackage{enumitem}
\usepackage{mathrsfs}
\usepackage[driver=pdftex]{geometry}
\usepackage{import}
%\usepackage{titleformat{\section
%        {\normalfont\normalzie\bfseries}{Helo.}{1em}{}


\definecolor{codegreen}{rgb}{0,0.6,0}
\definecolor{codegray}{rgb}{0.5,0.5,0.5}
\definecolor{codepurple}{rgb}{0.58,0,0.82}
\definecolor{backcolour}{rgb}{0.95,0.95,0.92}

\crefname{table}{table}{table}
\setlength{\parindent}{0em}
\setlength{\parskip}{0.7em}

\counterwithin{table}{section}
 
\lstdefinestyle{mystyle}{
    backgroundcolor=\color{backcolour},   
    commentstyle=\color{codegreen},
    keywordstyle=\color{magenta},
    numberstyle=\tiny\color{codegray},
    stringstyle=\color{codepurple},
    basicstyle=\ttfamily\footnotesize,
    breakatwhitespace=false,         
    breaklines=true,                 
    captionpos=b,                    
    keepspaces=true,                 
    numbers=left,                    
    numbersep=5pt,                  
    showspaces=false,                
    showstringspaces=false,
    showtabs=false,                  
    tabsize=2
}

\newtheorem{theorem}{Theorem}
\newtheorem{definition}{Definition}
\newtheorem{proof}{Proof}
 
\lstset{style=mystyle}

%\usepackage[margin=0.5in]{geometry}
\usepackage{inputenc}

\newcommand{\Real}{\mathbb{R}}
\newcommand{\Int}{\mathbb{Z}}
\newcommand{\Nat}{\mathbb{N}}
\newcommand{\Complex}{\mathbb{C}}
\newcommand{\vect}[1]{\boldsymbol{#1}}

%\renewcommand{\TPTminimum}{\textwidth}

\renewcommand{\Re}[1]{\mathfrak{Re}\left\lbrace{#1}\right\rbrace}
\renewcommand{\Im}[1]{\mathfrak{Im}\left\lbrace{#1}\right\rbrace}

%\DeclareMathOperator*{\minimize}{minimize}
%\DeclareMathOperator*{\maximize}{maximize}

\title{{\bf MATH 3172 3.0\\ Combinatorial Optimization}\\\vspace{10pt} \large Workshop 5     
    \author{Jacques Nel}
}

\begin{document}

\maketitle

\thispagestyle{empty}

\newpage

\pagenumbering{arabic}

\section*{A quick note about revision:}

This is the $3^\mathrm{rd}$ revision of this assignment.

\newpage


\section{A-4 Cane sugar production}

This problem is taken from\footnote{C. Guéret, C. Prins, M. Sevaux, \textit{Applications of optimization with Xpress-MP}. %
Paris: Dash Optimization Ltd., 2007. Page 74.}.

\subsection{Parameters}

Let $W = \left\lbrace 1,\ldots, m\right\rbrace$ enumerate $m=11$ wagons or lots, and 
$S=\left\lbrace 1, \ldots, n\right\rbrace$ enumerate $n$ time slots. The refinery has $k=3$ equivalent processing lines.
$n$ timeslots are required to process $m$ wagons where $n$ is given by $n=\texttt{ceil}(m/k)=4$. 
Each lot $w\in W$ has an associated hourly loss $\Delta_w$ and remaining lifespan $l_w$ until total loss.
Furthermore, a single lot takes $D=2$ hours to process on any given production line.

\subsection{Decision variable}

Let $\vect{x} = \left[ x_{ws}\right] \in \left\lbrace 0, 1\right\rbrace^{m\times n}$
where for $w\in W$ and $s\in S$,

$$x_{ws} = \begin{cases}
    1 & \text{lot }w\text{ is processed in slot }s\\
    0 & \text{otherwise}
\end{cases}.
$$
\subsection{Model}

We seek to minimize the loss in raw material resulting from fermentation due
to delayed processing of a lot. The model is

\begin{mini!}
    {\vect{x}}{f\left(\vect{x}\right)=\sum_{w\in W}\sum_{s\in S} sd\Delta_{w}x_{ws} \protect\label{eq:a4-obj}}{\label{eq:a4}}{}
    \addConstraint{\sum_{s\in S}x_{ws}}{=1, \forall w\in W \protect\label{eq:a4-cstr1}}
    \addConstraint{\sum_{w\in W}x_{ws}}{\leq k, \forall s\in S \protect\label{eq:a4-cstr2}}
    \addConstraint{\sum_{s\in S}sx_{ws}}{\leq l_w / d, \forall w\in W \protect\label{eq:a4-cstr3}}
\end{mini!}

The objective function \cref{eq:a4-obj} is the total loss in raw material resulting from delayed processing
summed over all lots and wagons. All lots must be assigned to exactly one slot as enforced by \cref{eq:a4-cstr1}.
Next, \cref{eq:a4-cstr2} guarantees that at most $k=3$ lots can be processed in any one timeslot. Finally,
\cref{eq:a4-cstr3} ensures that a lot is processed before its total loss occurs. Observe that total loss of a lot
occurs after $l_w / d$ slots.

\newpage

\subsection{Results}

The optimal solution results in a loss of $f\left(\vect{x}^*\right) = 1620$ kg with the following
time slot assignments:

\begin{table}[h]
    \centering
    \caption{Optimal time slot allocations for each lot}\label{table:a4-results}
    \begin{tabular}{cccc}
        \hline
        \textbf{Slot 1} & \textbf{Slot 2} & \textbf{Slot 3} & \textbf{Slot 4} \\
        \hline
        lot 3   & lot 1 & lot 10 & lot 2 \\
        lot 6   & lot 5 & lot 8 & lot 4 \\
        lot 7   & lot 8 & &\\
        \hline

    \end{tabular}

    \medskip
    \emph{Note:} Column $j$ is generated with $\left\lbrace w : x_{wj} = 1, \forall w \in
            \mathrm{W}\right\rbrace$.
\end{table}

\section{A-6 Production of electricity}

This problem is taken from\footnote{C. Guéret, C. Prins, M. Sevaux, \textit{Applications of optimization with Xpress-MP}. %
Paris: Dash Optimization Ltd., 2007. Page 78.}.

\subsection{Parameters}

Let $T=\lbrace 1,\ldots, n\rbrace$ enumerate $n=7$ time periods (of varying length) and $P=\lbrace 1,\ldots, m\rbrace$ enumerate
$m=4$ generator types. For a time period $t\in T$ let $l_t$
denote the length of time period in hours, and
let $d_t$ denote the forecasted power demand given by $d_t$.

\begin{table}[h]
    \center
    \caption{Length and forecasted demand of time periods}\label{table:a6-periods}
    \begin{tabular}{c c c}
        \hline
        \textbf{Period} $t$ & \textbf{Length} $l_t$ & \textbf{Demand} $d_t$ \\
        \hline
        1 & 6 & $1.2 \times 10^4$ \\
        2 & 3 & $3.2 \times 10^4$ \\
        3 & 3 & $2.5 \times 10^4$ \\
        4 & 2 & $3.6 \times 10^4$ \\
        5 & 4 & $2.5 \times 10^4$ \\
        6 & 4 & $3.0 \times 10^4$ \\
        7 & 2 & $1.8 \times 10^4$ \\
        \hline
    \end{tabular}
\end{table}

For each generator type $p\in P$, there are $a_p$ units available. Each unit has
a minimum base power output $\theta_p$ (if it is running) and can scale up to a
maximum output denoted $\psi_p$, but incurring additional operating cost.

\begin{table}[h]
    \center
    \caption{Number of available units and power output capacity}\label{table:a6-generators}
    \begin{tabular}{cccc}
        \hline
        \textbf{Type} & \textbf{Num. available} $a_p$ & \textbf{Min. output} $\theta_p$ & \textbf{Max output} $\psi_p$ \\
        \hline
        1 & 10 & $7.5\times 10^2$ & $1.75\times 10^3$ \\
        2 & 4 & $1.0\times 10^3$ & $1.5\times 10^3$ \\
        3 & 8 & $1.2\times 10^3$ & $2.0\times 10^3$ \\
        4 & 3 & $1.8\times 10^3$ & $3.5\times 10^3$ \\
        \hline
    \end{tabular}
\end{table}

\newpage

Starting a generator unit of type $p\in P$ incurs a startup cost $\lambda_p$. Running the generator
incurs a fixed cost per hour $\mu_p$. Additional scalable output, on top of
the base output, incurs a hourly cost $\nu_p$ that is proportional to the additional output.

\begin{figure}[h]
    \protect\label{fig:a6-costs}
    \center
    \caption{Various costs associated with generator types}
    \begin{tabular}{cccc}
        \hline
        \textbf{Type} & \textbf{Start cost} $\lambda_p$ & \textbf{Run cost} $\mu_p$ & \textbf{Add. cost} $\nu_p$ \\
        \hline
        1 & 5000 & 2250 & 2.7 \\
        2 & 1600 & 1800 & 2.2 \\
        3 & 2400 & 3750 & 1.8 \\
        4 & 1200 & 4800 & 3.8 \\
        \hline
    \end{tabular}
\end{figure}


\subsection{Decision variables}

Suppose $p\in P$ and $t\in T$. Let
$0 \leq x_{pt} \in\Int$ denote the number of generators of type $p$ started in
period $t$, and
$0 \leq y_{pt} \in\Int$ be the number of generators of type $p$ running in
period $t$. Finally, let $0 \leq z_{pt}\in\Real$ denote the additional power generated by unit of type $p$ during
period $t$. 

To simplify notation, let $\vect{x} = \left[x_{pt}\right] \in
\Int^{m\times n}, \vect{y} = \left[y_{pt}\right] \in \Int^{m\times n}$,
and $\vect{z} = \left[z_{pt}\right] \in \Real^{m\times n}$.

\subsection{Model}

\begin{mini!}
    {\vect{x}, \vect{y}, \vect{z}}{\sum_{t\in T}\sum_{p\in P} \lambda_{p}x_{pt} + l_t\left(\mu_p y_{pt} + \nu_p z_{pt}\right) \protect\label{eq:a6-obj}}{\label{eq:a6}}{}
    \addConstraint{x_{p1}}{\geq y_{p1} - y_{pn}, \forall p\in P \protect\label{eq:a6-cstr1}}
    \addConstraint{x_{pt}}{\geq y_{pt} - y_{p(t-1)}, \forall p\in P, 1 < t\in T \protect\label{eq:a6-cstr2}}
    \addConstraint{z_{pt}}{\leq \left(\psi_p - \theta_p\right)y_{pt}, \forall (p, t)\in P\times T \protect\label{eq:a6-cstr3}}
    \addConstraint{\sum_{p\in P} \theta_p y_{pt} + z_{pt}}{\geq d_t, \forall t\in T \protect\label{eq:a6-cstr4}}
    \addConstraint{\sum_{p\in P} \psi_p y_{pt}}{\geq 1.2 d_t, \forall t\in T \protect\label{eq:a6-cstr5}}
    \addConstraint{y_{pt}}{\leq a_p, \forall (p,t)\in P\times T \protect\label{eq:a6-cstr6}}
    \addConstraint{x_{pt}}{\geq 0, \forall (p,t)\in P\times T \protect\label{eq:a6-cstr7}}
    \addConstraint{y_{pt}}{\geq 0, \forall (p,t)\in P\times T \protect\label{eq:a6-cstr8}}
    \addConstraint{z_{pt}}{\geq 0, \forall (p,t)\in P\times T \protect\label{eq:a6-cstr9}}
\end{mini!}

The cost function \cref{eq:a6-obj} is simply the startup cost, running cost, and
additional power cost summed over all decision variables. The number of generators started
for $t=1$ is related to the number of generators running by \cref{eq:a6-cstr1}. This relationship
depends on the numbers generators running at the end period $t=n$. The next family of 
constraints \cref{eq:a6-cstr2} is similar to the above, but deals with the relationship
for $1< t \leq n$.

Additional power output $z_{pt}$ is bounded by the difference between the maximum and
the base output, ie. $\psi_p - \theta_p$, as expressed
by \cref{eq:a6-cstr3}. Next, \cref{eq:a6-cstr4} ensures that the total ouput of all
generator units meets forecasted demand $d_t$ for all $t\in T$. Furthermore, a $20\%$
safety buffer is required at all times. \Cref{eq:a6-cstr5} ensures that, if demand
were to suddenly spike, a minimum of $20\%$ of $d_t$ of additional capacity can instantly be 
made available, by increasing additional output $z_{pt}$ up its maximum $\psi_p-\theta_p$.

The family of constraints \cref{eq:a6-cstr6} simply places an upper bound on the number
of units running, in a given period, equal to the given available number of units
$a_p$ of each type. Finally, \cref{eq:a6-cstr7}, \cref{eq:a6-cstr8}, and
\cref{eq:a6-cstr9} simply enforce the canonical non-negativity of $x_{pt}$, $y_{pt}$, and $z_{pt}$
respectively.

\subsection{Results}

The optimal solution was found with a total operating cost of
$f\left(\vect{x}^*, \vect{y}^*, \vect{z}^*\right) = \$1,456,810$.

\begin{table}[ht]
    \centering
    \caption{Optimal power generation schedule for 4 generator types over 7 planning periods}
\begin{tabular}{clrrrrrrr}
    \hline
    \textbf{Type} & & \textbf{1} & \textbf{2} & \textbf{3} & \textbf{4} & \textbf{5} & \textbf{6} & \textbf{7} \\
\hline
    \textbf{1} & No. used & 3 & 4 & 4 & 7 & 3 & 3 & 3\\
 & Tot. output & 2250 & 4600 & 3000 & 8600 & 2250 & 2600 & 2250\\
 & Add. output & 0 & 1600 & 0 & 3350 & 0 & 350 & 0\\
 \hline
    \textbf{2} & No. used & 4 & 4 & 4 & 4 & 4 & 4 & 4\\
 & Tot. output & 5750 & 6000 & 4200 & 6000 & 4950 & 6000 & 5950\\
 & Add. output & 1750 & 2000 & 200 & 2000 & 950 & 2000 & 1950\\
 \hline
    \textbf{3} & No. used & 2 & 8 & 8 & 8 & 8 & 8 & 4\\
 & Tot. output & 4000 & 16000 & 16000 & 16000 & 16000 & 16000 & 8000\\
 & Add. output & 1600 & 6400 & 6400 & 6400 & 6400 & 6400 & 3200\\
 \hline
    \textbf{4} & No. used & 0 & 3 & 1 & 3 & 1 & 3 & 1\\
 & Tot. output & 0 & 5400 & 1800 & 5400 & 1800 & 5400 & 1800\\
 & Add. output & 0 & 0 & 0 & 0 & 0 & 0 & 0\\
 \hline
\end{tabular}

\medskip
\emph{Note:} The above table was generated from the solution values $x_{pt}^*, 
y_{pt}^*$, and $z_{pt}^*$ with \texttt{a-6\_report.py}.
\end{table}

\section{C-2 Production of drinking glasses}

This problem is taken from\footnote{C. Guéret, C. Prins, M. Sevaux, \textit{Applications of optimization with Xpress-MP}. %
Paris: Dash Optimization Ltd., 2007. Page 106.}.

\subsection{Parameters}

Let $W =\left\lbrace 1,\ldots, n\right\rbrace$ enumerate $n=12$ week-long planning
periods, and let $P=\left\lbrace 1,\ldots, m\right\rbrace$ enumerate the $m=6$
product variants. Each product $p\in P$ has a predicted demand $d_{pt}$ during week
$t\in W$ as given in \cref{table:c2-demands}.

\begin{table}[h]
    \center
    \caption{Predicted weekly demand for each product variant}\label{table:c2-demands}
    \begin{tabular}{c cccccccccccc}
        \hline
        \textbf{Week} & \textbf{1} &\textbf{2} &\textbf{3} &\textbf{4} &\textbf{5} &\textbf{6} &\textbf{7} &\textbf{8} &\textbf{9} &\textbf{10} &\textbf{11} &\textbf{12} \\
        \hline
        \textbf{V1} & 20 & 22 & 18 & 35 & 17 & 19 & 23 & 20 & 29 & 30 & 28 & 32 \\
        \textbf{V2} & 17 & 19 & 23 & 20 & 11 & 10 & 12 & 34 & 21 & 23 & 30 & 12 \\
        \textbf{V3} & 18 & 35 & 17 & 10 & 9  & 21 & 23 & 15 & 10 & 0 & 13 & 17 \\
        \textbf{V4} & 31 & 45 & 24 & 38 & 41 & 20 & 19 & 37 & 28 & 12 & 30 & 37 \\
        \textbf{V5} & 23 & 20 & 23 & 15 & 10 & 22 & 18 & 30 & 28 & 7 & 15 & 10 \\
        \textbf{V6} & 22 & 18 & 20 & 19 & 18 & 35 & 0 & 28 & 12 & 30 & 21 & 23 \\
        \hline
    \end{tabular}
\end{table}

Each product variant $p\in P$ has an associated basic production cost $\lambda_p$
and an inventory storage cost $\mu_p$ incurred on product in inventory over
given period. Production requires a known amount worker labour time $\delta_p$,
machine time $\pi_p$, and production area $\gamma_p$. In every period there
is a limited amount of worker time $\Delta$, available machine time $\Pi$, and
production area $\Gamma$. Lastly, at the start of planning period there exists volume 
$I_p$ of item $p$ in the inventory, and it is required that there is $F_p$ of item
$p$ at the end of the planning period in the inventory. All parameters are given in
\cref{table:c2-params}.

\begin{table}[h]
    \center
    \caption{Given costs, production resources, and inventory of product variants}
    \label{table:c2-params}
    \begin{tabular}{cccccccc}
        \hline
        & \textbf{prod. cost} & \textbf{inv. cost}  & 
        \textbf{init. stock}  & \textbf{fin. stock}  & 
        \textbf{labour}  & \textbf{mach. time}  & 
        \textbf{area} \\
        \hline
        \textbf{V1} & 100 & 25 & 50 & 10 & 3 & 2 & 4 \\
        \textbf{V2} & 80 & 28 & 20 & 10 & 3 & 1 & 5 \\
        \textbf{V3} & 110 & 25 & 0 & 10 & 3 & 4 & 5 \\
        \textbf{V4} & 90 & 27 & 15 & 10 & 2 & 8 & 6 \\
        \textbf{V5} & 200 & 10 & 0 & 10 & 4 & 11 & 4 \\
        \textbf{V6} & 150 & 20 & 10 & 10 & 4 & 9 & 9 \\
        \hline
    \end{tabular}
\end{table}

\newpage

\subsection{Decision variables}

For a given product $p\in P$ and week $t\in W$ let
$0 \leq x_{pt} \in\Int$ denote the production volume. 
Also let $0\leq y_{pt}\in\Int$ denote the amount of product stored in the inventory
at the end of period $t$. 

To simplify notation let $\vect{x} = \left[ x_{pt} \right] \in \Int^{m\times x}$
and $\vect{y} = \left[ y_{pt} \right] \in \Int^{m\times x}$.

\subsection{Model}

\begin{mini!}
    {\vect{x}, \vect{y}}{f\left(\vect{x}, \vect{y}\right)=\sum_{p\in P}\sum_{t\in W} \lambda_p x_{pt} + \mu_p y_{pt} \protect\label{eq:c2-obj}}{\label{eq:c2}}{}
    \addConstraint{y_{p1}}{= I_p + x_{pt} - d_{pt}, \forall p\in P \protect\label{eq:c2-cstr1}}
    \addConstraint{y_{pt}}{= y_{p(t-1)} + x_{pt} - d_{pt}, \forall p\in P, t\in W \protect\label{eq:c2-cstr2}}
    \addConstraint{y_{pn}}{= F_p, \forall p\in P, \protect\label{eq:c2-cstr3}}
    \addConstraint{\sum_{p\in P} \delta_p x_{pt}}{\leq \Delta, \forall t\in W, \protect\label{eq:c2-cstr4}}
    \addConstraint{\sum_{p\in P} \pi_p x_{pt}}{\leq \Pi, \forall t\in W, \protect\label{eq:c2-cstr5}}
    \addConstraint{\sum_{p\in P} \gamma_p x_{pt}}{\leq \Gamma, \forall t\in W, \protect\label{eq:c2-cstr6}}
    \addConstraint{x_{pt}}{\geq 0 \forall p\in P, t\in W, \protect\label{eq:c2-cstr7}}
    \addConstraint{y_{pt}}{\geq 0 \forall p\in P, t\in W, \protect\label{eq:c2-cstr8}}
\end{mini!}

We seek to minimize total production cost. \Cref{eq:c2-obj} is a cost function which simply sums the total production and storage
costs over all decision variables.

\Cref{eq:c2-cstr1} and \cref{eq:c2-cstr2} states that the inventory at time $t$ is
equal the previous inventory plus the product minus the demand. \Cref{eq:c2-cstr1}
makes special consideration for the initial inventory $I_p$. \Cref{eq:c2-cstr3}
ensures that the final inventory for product $p$ is equal to $F_p$ at the end of the
planning period.

\Cref{eq:c2-cstr4}, \cref{eq:c2-cstr5} and \cref{eq:c2-cstr6} ensures that limited
production factors: worker time capacity $\Delta$, machine time capacity 
$\Pi$, and production area $\Gamma$ constraints are respected. For example,
production of product $p$ during period $t$ requires $\pi_p x_{pt}$ machine hours, the total
of which shall not exceed $\Pi$ for the given period.

Lastly \cref{eq:c2-cstr7} and \cref{eq:c2-cstr8} are simply the cononical non-negativity
constraints on both decision variables $x_{pt}$ and $y_{pt}$.

\newpage

\subsection{Results}

A optimal solution $\left(\vect{x}^*, \vect{t}^*\right)$ is found with a total 
production cost of $f\left(\vect{x}^*, \vect{y}^*\right) = \$186,076$.

\begin{table}[h]
    \center
    \caption{Production and storage quantities for each product type}
\begin{tabular}{cccccccccccccc}
    \hline
    & \textbf{Week} & \textbf{1} &\textbf{2} &\textbf{3} &\textbf{4} &\textbf{5} &\textbf{6} &\textbf{7} &\textbf{8} &\textbf{9} &\textbf{10} &\textbf{11} &\textbf{12} \\
    \hline
    \textbf{1} & Prod. & 0 & 0 & 11 & 34 & 29 & 7 & 23 & 21 & 29 & 29 & 29 & 41\\
& Store & 30 & 8 & 1 & 0 & 12 & 0 & 0 & 1 & 1 & 0 & 1 & 10\\
\textbf{2} & Prod. & 7 & 21 & 14 & 17 & 11 & 10 & 12 & 34 & 21 & 23 & 30 & 22\\
& Store & 10 & 12 & 3 & 0 & 0 & 0 & 0 & 0 & 0 & 0 & 0 & 10\\
\textbf{3} & Prod. & 18 & 35 & 17 & 11 & 8 & 21 & 23 & 15 & 10 & 0 & 13 & 27\\
& Store & 0 & 0 & 0 & 1 & 0 & 0 & 0 & 0 & 0 & 0 & 0 & 10\\
\textbf{4} & Prod. & 16 & 45 & 24 & 38 & 41 & 20 & 20 & 36 & 29 & 11 & 31 & 46\\
& Store & 0 & 0 & 0 & 0 & 0 & 0 & 1 & 0 & 1 & 0 & 1 & 10\\
\textbf{5} & Prod. & 47 & 16 & 34 & 14 & 23 & 24 & 43 & 0 & 26 & 4 & 0 & 0\\
& Store & 24 & 20 & 31 & 30 & 43 & 45 & 70 & 40 & 38 & 35 & 20 & 10\\
\textbf{6} & Prod. & 14 & 17 & 20 & 18 & 18 & 35 & 1 & 27 & 12 & 49 & 28 & 7\\
& Store & 2 & 1 & 1 & 0 & 0 & 0 & 1 & 0 & 0 & 19 & 26 & 10\\
\hline
\end{tabular}
\end{table}

\emph{Note:} I choose to solve this as an integer programming model. This is
the reason why my results differ slightly from the book. The above table simply states the optimal
solution values for $x_{pt}^*$ and $y_{pt}^*$ for all $p\in P$ and $t\in W$.

\section{D-5 Cutting sheet metal}

This problem is taken from\footnote{C. Guéret, C. Prins, M. Sevaux, \textit{Applications of optimization with Xpress-MP}. %
Paris: Dash Optimization Ltd., 2007. Page 134.}.

\subsection{Parameters}

Let $S =\left\lbrace 1, \ldots, n\right\rbrace$ enumerate $n=4$ different sizes, ie.
$\left\lbrace \texttt{36x50}, \texttt{24x36}, \texttt{20x60}, \texttt{18x30}\right\rbrace$.
Also, let $P=\left\lbrace 1,\ldots, m\right\rbrace$ enumerate $m$ different cutting patterns.
For $s\in S$ and $p\in P$ let $c_{sp}$ denote number of pieces of size $s$ yielded
by pattern $p$. The values of $c_{sp}$ are given by \cref{table:d5-yield}.

\begin{table}[h]
    \center
    \caption{Yields of various cutting patterns}\label{table:d5-yield}
    \begin{tabular}{ccccccccccccccccc}
        \hline
        \textbf{Pattern} & \textbf{1} & \textbf{2} &\textbf{3} &\textbf{4} &\textbf{5}&\textbf{6}&\textbf{7}&\textbf{8}&\textbf{9}&\textbf{10}&\textbf{11}&\textbf{12}&\textbf{13}&\textbf{14}&\textbf{15}&\textbf{16} \\
        \hline
        \texttt{36x50} & 1 & 1 & 1 & 0 & 0 & 0 &0 & 0 & 0 & 0 & 0 & 0 & 0 & 0 & 0 & 0 \\
        \texttt{24x36} & 2 & 1 & 0 & 2 & 1 & 0 & 3 & 2 & 1 & 0 & 5 & 4 & 3 & 2 & 1 & 0 \\
        \texttt{20x60} & 0 & 0 & 0 & 2 & 2 & 2 & 1 & 1 & 1 & 1 & 0 & 0 & 0 & 0 & 0 & 0 \\
        \texttt{18x30} & 0 & 1 & 3 & 0 & 1 & 3 & 0 & 2 & 3 & 5 & 0 & 1 & 3 & 5 & 6 & 8 \\
        \hline
    \end{tabular}
\end{table}

Each cut size $s\in S$ has a given demand $\vect{d} =\left( d_s : s \in S\right)^T
= \left( 8, 13, 5, 15\right)^T$. Finaly, each pattern has equivalent cost $\kappa = 1$,
or simply the cost of each sheet of raw material.

\subsection{Decision variable}

Let $0 \leq x_{p} \in \Int$ denote the number of times pattern $p$ is used. To simplify
notation let $\vect{x} = \left( x_p : p \in P\right)$.

\subsection{Model}

\begin{mini!}
    {\vect{x}}{f(\vect{x}) = \sum_{p\in P} x_p\kappa \protect\label{eq:d5-obj}}{\label{eq:d5}}{}
    \addConstraint{\sum_{p\in P} c_{sp} x_p}{\geq d_s, \forall s\in S \protect\label{eq:d5-cstr1}}
    \addConstraint{x_p}{\geq 0, \forall p\in P \protect\label{eq:d5-cstr2}}
\end{mini!}

We seek to minimize the total cost which is given by \cref{eq:d5-obj}. This is simply
the total number of sheets of raw material used. \Cref{eq:d5-cstr1} is the family
of demand constraints, which guarantee that demand is met for each size $s\in S$.
Finaly \cref{eq:d5-cstr2} is simply the canonical non-negativity constraint on
the decision variable $x_p$.

\subsection{Results}

An optimal solution is found which uses 11 sheets of raw material to satisfy
demand, with a cost function value of $f\left(\vect{x}^*\right)= 11$. The following quantities of each pattern
are used:

\texttt{pattern 1} = 3,
\texttt{pattern 3} = 5,
\texttt{pattern 4} = 2,
\texttt{pattern 7} = 1,
and all others are unused. These values are simply the optimal non-zero values of the decision variable
$x_p^*, \forall p\in\mathrm{P}$.

\section{F-1 Flight connections at a hub}

This problem is taken from\footnote{C. Guéret, C. Prins, M. Sevaux, \textit{Applications of optimization with Xpress-MP}. %
Paris: Dash Optimization Ltd., 2007. Page 157.}.

\subsection{Parameters}

Let $P = \left\lbrace 1, \ldots, n\right\rbrace$ enumerate both $n=6$ incoming flights
and then $n$ outgoing flights. For $i\in P$ and $j\in P$, let $\mu_{ij}$ denote
the number of passengers arriving on flight $i$ from origin $i$ seeking to continue
on to destination $j$ given by \cref{table:f1-passengers}.

\begin{table}[h]
    \center
    \caption{Arriving passengers and destinations}\label{table:f1-passengers}
    \begin{tabular}{c|cccccc}
        \hline
        \textbf{City} & \textbf{1} & \textbf{2} & \textbf{3} & \textbf{4} & \textbf{5} & \textbf{6} \\
        \textbf{1} & 35 & 12 & 16 & 38 & 5 & 2 \\ 
        \textbf{2} & 25 & 8 & 9 & 24 & 6 & 8 \\
        \textbf{3} & 12 & 8 & 11 & 27 & 3 & 2 \\
        \textbf{4} & 38 & 15 & 14 & 30 & 2 & 9 \\
        \textbf{5} & - & 9 &   8 & 25 &  10 & 5 \\
        \textbf{6} & - & - & - & 14 &  6 & 7 \\
        \hline
    \end{tabular}
\end{table}

\subsection{Decision variable}

Let $x_{ij} \in\lbrace 0, 1\rbrace$ indicate that aircraft
    from origin $i\in P$ travels to destination $j\in P$ for next flight when
    $x_{ij} = 1$. To simplify notation let $\vect{x} = \left[x_{ij}\right]
    \in \left\lbrace 0, 1\right\rbrace^{n\times n}$.

\subsection{Model}

We seek to minimize the number of passengers requiring to disembark and transfer to another plane for their
next flight. In other words, we wish to maximize the number of passengers staying on their arriving aircraft.

\begin{mini!}
    {\vect{x}}{f(\vect{x}) = \sum_{i\in P} \sum_{j\in P} \mu_{ij}x_{ij} \protect\label{eq:f1-obj}}{\label{eq:f1}}{}
    \addConstraint{\sum_{j\in P} \mu_{ij}x_{ij}}{= 1, \forall i\in P \protect\label{eq:f1-cstr1}}
    \addConstraint{\sum_{i\in P} \mu_{ij}x_{ij}}{= 1, \forall j\in P. \protect\label{eq:f1-cstr2}}
\end{mini!}

\subsection{Results}

The optimal solution has a total of $f\left(\vect{x}^*\right)= 112$ passengers remaining on their arrival flights for the
remainder of their journeys. The following assignment of aircraft to destination, minimizes passenger
inconvenience:
\medskip

Bordeaux $\rightarrow$ London 38

Clermon-Ferrand $\rightarrow$ Bern 8

Marseille $\rightarrow$ Brussels 11

Nantes $\rightarrow$ Berlin 38

Nice $\rightarrow$ Rome 10

Toulouse $\rightarrow$ Vienna 7

\medskip
\emph{Note:} The above mapping is simply constructed by the permutation matrix given by $\vect{x}^*$.

Refer to \texttt{f-1\_report.py}
for implementation details.

\end{document}
