% ------------------
% -- Assignment 1 --
% -- Math 4171 -----

\documentclass[11pt,oneside]{article}

%\usepackage{subfigure}
\usepackage{subcaption}
\usepackage{graphicx}
\usepackage{amsmath}
\usepackage{amsfonts}
\usepackage{hyperref}
\usepackage{adjustbox}
\usepackage{listings}
\usepackage{xcolor}
\usepackage{titlesec}
\usepackage{enumitem}
\usepackage{mathrsfs}
\usepackage[driver=pdftex]{geometry}
\usepackage{import}
%\usepackage{titleformat{\section
%        {\normalfont\normalzie\bfseries}{Helo.}{1em}{}


\definecolor{codegreen}{rgb}{0,0.6,0}
\definecolor{codegray}{rgb}{0.5,0.5,0.5}
\definecolor{codepurple}{rgb}{0.58,0,0.82}
\definecolor{backcolour}{rgb}{0.95,0.95,0.92}
 
\lstdefinestyle{mystyle}{
    backgroundcolor=\color{backcolour},   
    commentstyle=\color{codegreen},
    keywordstyle=\color{magenta},
    numberstyle=\tiny\color{codegray},
    stringstyle=\color{codepurple},
    basicstyle=\ttfamily\footnotesize,
    breakatwhitespace=false,         
    breaklines=true,                 
    captionpos=b,                    
    keepspaces=true,                 
    numbers=left,                    
    numbersep=5pt,                  
    showspaces=false,                
    showstringspaces=false,
    showtabs=false,                  
    tabsize=2
}

\newtheorem{theorem}{Theorem}
\newtheorem{definition}{Definition}
\newtheorem{proof}{Proof}
 
\lstset{style=mystyle}

%\usepackage[margin=0.5in]{geometry}
\usepackage{inputenc}

\newcommand{\Real}{\mathbb{R}}
\newcommand{\Int}{\mathbb{Z}}
\newcommand{\Nat}{\mathbb{N}}
\newcommand{\Complex}{\mathbb{C}}
\newcommand{\vect}[1]{\boldsymbol{#1}}

\renewcommand{\Re}[1]{\mathfrak{Re}\left\lbrace{#1}\right\rbrace}
\renewcommand{\Im}[1]{\mathfrak{Im}\left\lbrace{#1}\right\rbrace}

\DeclareMathOperator*{\minimize}{minimize}
\DeclareMathOperator*{\maximize}{maximize}

\title{{\bf LE/EECS 3172 3.0\\ Combinatorial Optimization}\\\vspace{10pt} \large Workshop 5     
    \author{Jacques Nel}
}

\begin{document}

\maketitle

\newpage

\section{A-4 Cane sugar production}

\begin{table}[h]
    \center
    \begin{tabular}{lccccccccccc}
        \hline
        \textbf{Lot} & 1 & 2 & 3 & 4 & 5 & 6 & 7 & 8 & 9 & 10 & 11 \\
        \hline
        \textbf{Loss (kg/h)} & 43 & 26 & 37 & 28 & 13 & 54 & 62 & 49 & 19 & 28 & 30 \\
            \textbf{Life span (h)} & 8 & 8 & 2 & 8 & 4 & 8 & 8 & 8 & 6 & 8 & 8 \\
        \hline
    \end{tabular}
\end{table}

\subsection{Parameters}

Let $W = \lbrace 0,\ldots, m-1\rbrace$ enumerate the wagons or lots. Let
$S = \lbrace 0, \ldots, n-1\rbrace$ enumerate the time slots. Let $k=3$ denote
the number of processing lines in the refinery, with $m=11$ and
$n=\texttt{ceil}(m/k) = 4$.

For $w\in W$ and $s\in S$, let $\Delta_{w}$ denote the hourly loss given in the
table above for lot $w$. Let $l_{w}$ denote the life span of lot $w$. Let
$d=2$ denote the time it takes to process one lot.

\subsection{Decision variables}

For $(w, s) \in W\times S$, let $x_{ws}\in\lbrace 0, 1\rbrace$ be a binary
decision variable equal to 1 if lot $w$ is processed in slot $s$, and 0
otherwise.

\subsection{Model}

We seek to minimize the loss in raw material resulting from fermentation due
to delayed processing of a lot. The model is

$$
\minimize \sum_{w\in W}\sum_{s\in S} (s+1)d \Delta_w x_{ws},
$$

subject to the following constraints:

$$
\sum_{s\in S} x_{ws} = 1, \forall w \in W,
$$
$$
\sum_{w\in W} x_{ws} \leq k, \forall s \in S,
$$
$$
\sum_{s\in S} (s+1)x_{ws} \leq \frac{l_w}{d}, \forall w \in W.
$$

\subsection{Results}

The optimal solution results in a loss of 1620 kg with the following
time slot assignments:

\begin{table}[h]
    \center
    \begin{tabular}{cccc}
        \hline
        \textbf{Slot 1} & \textbf{Slot 2} & \textbf{Slot 3} & \textbf{Slot 4} \\
        \hline
        lot 3   & lot 1 & lot 10 & lot 2 \\
        lot 6   & lot 5 & lot 8 & lot 4 \\
        lot 7   & lot 8 \\

    \end{tabular}
\end{table}

\section{A-6 Production of electricity}

\subsection{Parameters}

Let $L=\left(l_0, \ldots, l_{n-1}\right)$ be the length of the time periods, and
let $D=\left(d_0, \ldots, d_{n-1}\right)$ be the forecasted demand for each
period. Let $T = \lbrace 0, \ldots, n-1\rbrace$ enumerate the time periods.
Let $P=\lbrace 0, \ldots, m-1\rbrace$ with $m=4$ enumerate the types of generators.

For a generator variant $p\in P$, a particular type has minimum base power output
when it is running, along with a maximum output capacity. Let these parameters
be denoted by $\phi_p$ and $\psi_p$ respectively. A generator also the following costs
associated with it:

$$
\lambda_p : \textrm{startup cost}
\quad\quad
\mu_p : \textrm{running cost}
\quad\quad
\nu_p : \textrm{scalable additional cost}
$$

Lastly, the available number of each type of generator $p$ is denoted by $a_p$.

\subsection{Decision variables}

For $(p, t) \in P \times T$,

$0 \leq i_{pt}\in\Nat$ : number of generators of type $p$ started in period $t$

$0 \leq j_{pt}\in\Nat$ : number of generators of type $p$ running in period $t$

$0 \leq q_{pt}\in\Real$ : additional scalable power output from generator $p$ in period $t$.

\subsection{Model}

The objective is to simply minimize total operating cost.

$$
\minimize \sum_{t\in T}\sum_{p\in P}
\lambda_p i_{pt} + l_t\left( \mu_p j_{pt} + \nu_{p} q_{pt} \right),
$$

subject to the following constraints:

$$
i_{p0} \geq j_{p0} - j_{p(n-1)}, \forall p \in P
$$
$$
i_{pt} \geq j{pt} - j_{p(t-1)}, \forall p \in P, 0 < t \in T
$$
$$
q_{pt} \leq \left(\psi_p - \phi_p\right)j_{pt}, \forall (p, t) \in P\times T
$$
$$
\sum_{p\in P} \phi_p j_{pt} + q_{pt} \geq d_t, \forall t \in T
$$
$$
\sum_{p\in P} \psi_p j_{pt} \geq 1.2 d_t, \forall t \in T
$$
$$
j_{pt} \leq a_{p}, \forall (p, t) \in P \times T
$$
$$
i_{pt} \geq 0, j_{pt} \geq 0, q_{pt} \geq 0 
$$

\subsection{Results}

The optimal solution was found with a total operating cost of \$1,456,810.

\begin{table}[h]
    \center
\begin{tabular}{clrrrrrrr}
    \hline
    \textbf{Type} & & \textbf{1} & \textbf{2} & \textbf{3} & \textbf{4} & \textbf{5} & \textbf{6} & \textbf{7} \\
\hline
    \textbf{1} & No. used & 3 & 4 & 4 & 7 & 3 & 3 & 3\\
 & Tot. output & 2250 & 4600 & 3000 & 8600 & 2250 & 2600 & 2250\\
 & Add. output & 0 & 1600 & 0 & 3350 & 0 & 350 & 0\\
 \hline
    \textbf{2} & No. used & 4 & 4 & 4 & 4 & 4 & 4 & 4\\
 & Tot. output & 5750 & 6000 & 4200 & 6000 & 4950 & 6000 & 5950\\
 & Add. output & 1750 & 2000 & 200 & 2000 & 950 & 2000 & 1950\\
 \hline
    \textbf{3} & No. used & 2 & 8 & 8 & 8 & 8 & 8 & 4\\
 & Tot. output & 4000 & 16000 & 16000 & 16000 & 16000 & 16000 & 8000\\
 & Add. output & 1600 & 6400 & 6400 & 6400 & 6400 & 6400 & 3200\\
 \hline
    \textbf{4} & No. used & 0 & 3 & 1 & 3 & 1 & 3 & 1\\
 & Tot. output & 0 & 5400 & 1800 & 5400 & 1800 & 5400 & 1800\\
 & Add. output & 0 & 0 & 0 & 0 & 0 & 0 & 0\\
 \hline
\end{tabular}
\end{table}

\section{C-2 Production of drinking glasses}

\begin{table}[h]
    \center
    \caption{Demands for the planning periods (batches of 1000 glasses)}
    \begin{tabular}{ccccccccccccc}
        \hline
        \textbf{Week} & \textbf{1} & \textbf{2} & \textbf{3} & \textbf{4} & \textbf{5} & \textbf{6} & \textbf{7} & \textbf{8} & \textbf{9} & \textbf{10} & \textbf{11} & \textbf{12} \\
        \hline
        \textbf{V1} & 20 & 22 & 18 & 35 & 17 & 19 & 23 & 20 & 29 & 30 & 28 & 32 \\
        \textbf{V2} & 17 & 19 & 23 & 20 & 11 & 10 & 12 & 34 & 21 & 23 & 30 & 12 \\
        \textbf{V3} & 18 & 35 & 17 & 10 & 9  & 21 & 23 & 15 & 10 & 0  & 13 & 17 \\
        \textbf{V4} & 31 & 45 & 24 & 38 & 41 & 20 & 19 & 37 & 28 & 12 & 30 & 37 \\
        \textbf{V5} & 23 & 20 & 23 & 15 & 10 & 22 & 18 & 30 & 28 & 7 & 15 & 10 \\
        \textbf{V6} & 22 & 18 & 20 & 19 & 18 & 35 & 0  & 28 & 12 & 30 & 21 & 23 \\
        \hline
    \end{tabular}
\end{table}

\begin{table}[h]
    \center
    \caption{Data for the six glass types}
    \begin{tabular}{cccccccc}
        \hline
        & prod. cost & store cost & init. stock & final stock & $W$ & $M$ & $S$ \\
        \hline
        \textbf{V1} & 100 & 25 & 50 & 10 & 3 & 2 & 4 \\
        \textbf{V2} & 80 & 28 & 20 & 10 & 3 & 1 & 5 \\
        \textbf{V3} & 110 & 25 & 0 & 10 & 3 & 4 & 5 \\
        \textbf{V4} & 90 & 27 & 15 & 10 & 2 & 8 & 6 \\
        \textbf{V5} & 200 & 10 & 0 & 10 & 4 & 11 & 4 \\
        \textbf{V6} & 150 & 20 & 10 & 10 & 4 & 9 & 9 \\
        \hline
    \end{tabular}
\end{table}

\subsection{Parameters}

Let $n=12$ and $m=6$ denote the number of planning periods and number of product
variants respectively. Let $W=\lbrace 0,\ldots,n-1\rbrace$ enumerate the planning
periods. Let $P=\lbrace 0,\ldots, m-1\rbrace$ enumerate the different products.

For $p\in P$ and $w\in W$, let

$d_{pw}$ : demand for product $p$ in week $w$,

$\lambda_{p}$ : production cost for each product,

$\mu_{p}$ : storage cost for each product,

$I_{p}$ : initial stock for each each product at start,

$F_{p}$ : final stock requirement for each product at end,

$\delta_p$ : worker time cost required per product,

$\pi_p$ : machine time cost required per product,

$\lambda_p$ : storage area required for production,

$\Delta = 390$ : available worker time capacity,

$\Pi = 850$ : available machine time capacity,

$\Lambda = 1000$ : available production storage area.

\subsection{Decision variables}

Let $0 \leq x_{pw} \in \Int$ denote the quanitity of production of product
$p$ in period $w$. Let $0 \leq y_{pw} \in \Int$ denote the storage of product
$p$ in period $w$.

\subsection{Model}

The objective is simply the total production cost, which we seek to minimize.

$$
\minimize \sum_{p\in P}\sum_{t \in T}
\lambda_p x_{pw} + \mu_p y_{pw}
$$

Subject to the following constraints:

\begin{eqnarray}
y_{p0} = I_p + x_{p0} - d_{p0}, \forall p \in P, \\
y_{pt} = y_{p(t-1)} + x_{pt} - d_{pt}, \forall p \in T \text{ and } 0 < t \in T \\
y_{p(n-1)} \geq F_p, \forall p \in P \\
\sum_{p\in P} \delta_p x_{pt} \leq \Delta, \forall t \in T\\
\sum_{p\in P} \pi_p x_{pt} \leq \Pi, \forall t \in T \\
\sum_{p\in P} \lambda_p x_{pt} \leq \Lambda, \forall t  \in T\\
x_{pt} \geq 0, y_{pt} \geq 0, \forall (p, t) \in P\times T
\end{eqnarray}

\subsection{Results}

A optimal solution is found with a total production cost of \$186,076.

\begin{table}[h]
    \center
    \caption{Production and storage quantities for each product type}
\begin{tabular}{cccccccccccccc}
    \hline
    & \textbf{Week} & \textbf{1} &\textbf{2} &\textbf{3} &\textbf{4} &\textbf{5} &\textbf{6} &\textbf{7} &\textbf{8} &\textbf{9} &\textbf{10} &\textbf{11} &\textbf{12} \\
    \hline
    \textbf{1} & Prod. & 0 & 0 & 11 & 34 & 29 & 7 & 23 & 21 & 29 & 29 & 29 & 41\\
& Store & 30 & 8 & 1 & 0 & 12 & 0 & 0 & 1 & 1 & 0 & 1 & 10\\
\textbf{2} & Prod. & 7 & 21 & 14 & 17 & 11 & 10 & 12 & 34 & 21 & 23 & 30 & 22\\
& Store & 10 & 12 & 3 & 0 & 0 & 0 & 0 & 0 & 0 & 0 & 0 & 10\\
\textbf{3} & Prod. & 18 & 35 & 17 & 11 & 8 & 21 & 23 & 15 & 10 & 0 & 13 & 27\\
& Store & 0 & 0 & 0 & 1 & 0 & 0 & 0 & 0 & 0 & 0 & 0 & 10\\
\textbf{4} & Prod. & 16 & 45 & 24 & 38 & 41 & 20 & 20 & 36 & 29 & 11 & 31 & 46\\
& Store & 0 & 0 & 0 & 0 & 0 & 0 & 1 & 0 & 1 & 0 & 1 & 10\\
\textbf{5} & Prod. & 47 & 16 & 34 & 14 & 23 & 24 & 43 & 0 & 26 & 4 & 0 & 0\\
& Store & 24 & 20 & 31 & 30 & 43 & 45 & 70 & 40 & 38 & 35 & 20 & 10\\
\textbf{6} & Prod. & 14 & 17 & 20 & 18 & 18 & 35 & 1 & 27 & 12 & 49 & 28 & 7\\
& Store & 2 & 1 & 1 & 0 & 0 & 0 & 1 & 0 & 0 & 19 & 26 & 10\\
\hline
\end{tabular}
\end{table}

\emph{Note:} I choose to solve this as an integer programming model. This is
the reason why my results differ slightly from the book.

\section{D-5 Cutting sheet metal}

\begin{table}[h]
    \center
    \caption{Summary of cutting patterns}
    \begin{tabular}{ccccccccccccccccc}
        \hline
        \textbf{Pattern} & \textbf{1}& \textbf{2}& \textbf{3}& \textbf{4}& \textbf{5}& \textbf{6}& \textbf{7}& \textbf{8}& \textbf{9}& \textbf{10}& \textbf{11}& \textbf{12}& \textbf{13}& \textbf{14}& \textbf{15}& \textbf{16} \\
        \hline
        \textbf{36 x 50} & 1 & 1 & 1 & 0 & 0 & 0 & 0 & 0 & 0 & 0 & 0 & 0 & 0 & 0 & 0 & 0 \\
        \textbf{24 x 36} & 2 & 1 & 0 & 2 & 1 & 0 & 3 & 2 & 1 & 0 & 5 & 4 & 3 & 2 & 1 & 0 \\
        \textbf{20 x 60} & 0 & 0 & 0 & 2 & 2 & 2 & 1 & 1 & 1 & 1 & 0 & 0 & 0 & 0 & 0 & 0 \\
        \textbf{18 x 30} & 0 & 1 & 3 & 0 & 1 & 3 & 0 & 2 & 3 & 5 & 0 & 1 & 3 & 5 & 6 & 8 \\
        \hline
    \end{tabular}
\end{table}

\subsection{Parameters}

Let $S =\lbrace 0, \ldots n-1\rbrace$ with $n=4$ enumerate the sizes
$\lbrace \texttt{36x50}, \texttt{24x36}, \texttt{20x60}, \texttt{18x30}\rbrace$.
Let $P=\lbrace 0, \ldots m-1\rbrace$ enumerate the cutting patterns with $m=16$.

For $s\in S$ and $p\in P$, let

$d_{s}$ : demand for size $s$,

$c_{sp}$ : from the table above; with yields of each pattern,

$K=1$ : the cost of each pattern; simply cost of invidual raw sheet material

\subsection{Decision variables}

Let $0 \leq u_p \in \Int$ be a integer decision variable denoting the number
of times pattern $p$ is used.

\subsection{Model}

The objective is simply the total raw material cost, ie. the number of sheets
of metal used.

$$
\minimize \sum_{p\in P} u_p K
$$

subject to the following constraints:

\begin{eqnarray}
    \sum_{p\in P} c_{sp}u_p \geq d_{s}, \forall s \in S \\
    u_p \geq 0, \forall p\in P
\end{eqnarray}

\subsection{Results}

An optimal solution is found which uses 11 sheets of raw material to satisfy
demand, with a objective function value of 11. The following quantities of each pattern
are used:

\texttt{pattern 1} = 3,
\texttt{pattern 3} = 5,
\texttt{pattern 4} = 2,
\texttt{pattern 7} = 1,
and all others are unused.

\section{F-1 Flight connections at a hub}

\subsection{Parameters}

Let $P = \lbrace 0, \ldots n-1\rbrace$ enumerate all the set of incoming flights
with $n=6$. $P$ enumerates both origins of incoming flights and flights to
destinations, since the number of aircraft is fixed.

Let $p_{ij}$ denote the number of passengers arriving from $i$ and having final
destination $j$.

\subsection{Decision variables}

Let $x_{ij}\in \lbrace 0, 1\rbrace$ be a binary variable, indicating plane
from city $i$ will depart to city $j$ next when 1.

\subsection{Model}

We simply wish to maximize the number of passengers that do not have to disembark
and transfer to another aircraft. Our objective function is

$$
\maximize \sum_{i\in P}\sum_{j\in P} p_{ij}x_{ij}
$$

subject to the following constraints:

$$
\sum_{j\in P}p_{ij} x_{ij} = 1, \forall i\in P
$$
$$
\sum_{i\in P}p_{ij} x_{ij} = 1, \forall j\in P
$$

\subsection{Results}

The optimal solution has a value of $112$ for its objective function.

The following assignment of aircraft to destination, minimizes passenger
inconvenience:

Bordeaux $\rightarrow$ London 38

Clermon-Ferrand $\rightarrow$ Bern 8

Marseille $\rightarrow$ Brussels 11

Nantes $\rightarrow$ Berlin 38

Nice $\rightarrow$ Rome 10

Toulouse $\rightarrow$ Vienna 7




\end{document}
