\section{From the text}

\subsection{Chapter 1.1}

If $z=x+iy$ is a complex number with $x,y\in\Real$, we define

$$
|z| =\left(x^2+y^2\right)^{1/2}
$$

and call this quantity the \textbf{modulus} or \textbf{absolute value} of $z$.

\begin{enumerate}[label=(\alph*)]

    \item What is the geometric interpretation of $|z|$?

        It is natural to interpret $|z|$ as the Euclidean $L_2$ distance of a vector
        
        $\vect{x} = \begin{pmatrix}\Re{z} \\ \Im{z}\end{pmatrix} \in\Real^2$

    \item Show that if $|z|=0$, then $z=0$.

        Suppose $z=a+bi$. $0=|z|=a^2+b^2 = 0$, but since $a,b\in\Real
        \implies a =0$ and $b=0$.

    \item Show that if $\lambda\in\Real$, then $|\lambda z|=|\lambda||z|$, where
        $|\lambda|$ denotes the standard absolute value of a real number.

        Suppose $\Complex\ni z = a+bi$ and $\lambda\in\Real$, then
        $|\lambda z| = |\lambda (a+bi)| = \sqrt{\lambda^2 a^2+\lambda^2 b^2}$

        $=|\lambda|\sqrt{a^2+b^2}=|\lambda||z|$.

    \item If $z_1$ and $z_2$ are two complex numbers, prove that

        $$
        |z_1z_2|=|z_1||z_2|
        \quad\quad\text{and}\quad\quad
        |z_1+z_2|\leq |z_1|+|z_2|.
        $$

        Let $z_1 = a+bi$ and $z_2 = c+di$.

        $$
        |z_1z_2|=|ac - bd + adi + cbi|
        =
        \sqrt{(ac-bd)^2+(ad+cb)^2}
        $$
        $$
        \sqrt{a^2+b^2}\sqrt{c^2+d^2}
        =|z_1||z_2|
        $$

        $$
        |z_1 + z_2| = |a+c + i(b+d)| = \sqrt{(a+c)^2 + (b+d)^2}
        $$
        $$
        =
        \sqrt{a^2+c^2+2ac + b^2 + d^2 + 2bd}
        \leq
        \sqrt{a^2+b^2} + \sqrt{c^2+d^2}
        =|z_1| + |z_2|
        $$

    \item Show that if $z\neq 0$, then $|1/z| = 1/|z|$.

        Suppose $z\neq 0$.
        $$
    |1/z| = \left|\frac{1}{z}\cdot\frac{\bar{z}}{\bar{z}}\right|
    =\left|
    \frac{a-bi}{(a+bi)(a-bi)}\right|=\frac{1}{|a+bi|}
        $$

\end{enumerate}

\newpage

\subsection{Chapter 1.2.c}

If $z=x+iy$ is a complex number with $x,y\in\Real$, we define the \textbf{complex conjugate}
of $z$ by

$$
\bar{z} = x-iy
$$

\begin{enumerate}[label=(\alph*)]

    \item[(c)] Prove that if $z$ belongs to the unit circle, then $1/z=\bar{z}$.

        Let $z=x+iy$. Suppose $z$ is on the unit circle, then

        $$
        |z| = \sqrt{x^2+y^2} = 1 
        $$
        $$
        1/z = \frac{(x+iy)(x-iy)}{x-iy}
        =\frac{|z|}{\bar{z}} = \frac{1}{\bar{z}}
        $$

\end{enumerate}

\subsection{Chapter 1.3.b}

A sequence of complex numbers $\lbrace w_n\rbrace^\infty_{n=1}$ is said to converge if
there exists $w\in\Complex$ such that

$$
\lim_{n\rightarrow\infty}|w_n-w| = 0,
$$

and we say that $w$ is the limit of the sequence.

The sequence $\lbrace w_n\rbrace^\infty_{n=1}$ is said to be a \textbf{Cauchy sequence}
if for every $\epsilon >0$ there exists a positive integer $N$ such that

$$
|w_n-w_m| < \epsilon
\quad\quad
\text{whenever }
n,m > N.
$$

\begin{enumerate}[label=(\alph*)]
    \item[(b)] Prove that a sequence of complex numbers converges if and only if it is a Cauchy
        sequence. [Hint: A similiar theorem exists for the convergence of a sequence of
        real numbers. Why does it carry over to sequences of complex numbers?]

        \begin{proof}
            Suppose $(w_k)$ converges to $w$. The triangle inequality gives us
            $$
            |w_n-w_m|=|w_n-w+w-w_m|\leq |w_n-w| + |w-w_m|.
            $$

            Thus, given $\forall \epsilon > 0$, choose $N\in\mathbb{N}$ such that
            $k\geq N$ implies that $|w_k-s| < \epsilon/2.$. Then for $m,n>N$, we have

            $$
            |w_n-w_m| \leq |w_n-w|+ |w-w_m| < \frac{\epsilon}{2}+\frac{\epsilon}{2}=\epsilon.
            $$

            Therefore $(w_k)$ is is a Cauchy sequence. $\square$.
        \end{proof}
\end{enumerate}

\newpage

\subsection{Chapter 1:4\{ c, d, e, f, g, i \}}

For $z\in\Complex$, we define the \textbf{complex exponential} by

$$
e^z = \sum_{n=0}^\infty \frac{z^n}{n!}.
$$

\begin{enumerate}[label=(\alph*)]

    \item[(c)] Show that if $z$ is purely imaginary, that is, $z=iy$ with $y\in\Real$, then

        $$
        e^{iy} = \cos y + i \sin y.
        $$

        This is Euler's identity. [Hint: Use power series.]

        $$
        e^{iy} =1 + \frac{iy}{1!} - \frac{y^2}{2!} - \frac{iy^3}{3!}
        + \frac{y^4}{4!} + \cdots
        $$

        Taking the $\Re{e^{iy}}$ part yields the power series for $\cos{y}$

        $$
        \cos{y} = 1 - \frac{y^2}{2!} + \frac{y^4}{4!} + \cdots
        $$

        Taking the $\Im{e^{iy}}$ part yields the power series for $\cos{y}$

        $$
        \sin{y} = \frac{y}{1!} - \frac{y^3}{3!} + \cdots
        $$

        Thus,

        $$
        e^{iy} = \cos{y} + i\sin{y}.\quad\square
        $$
    
    \item[(d)] More generally,

        $$
        e^{x+iy} = e^{x}\left(\cos{y} + i\sin{y}\right)
        $$

        whenever $x,y\in\Real$, and show that

        $$
        \left|e^{x+iy}\right| = e^{x}
        $$

        $$
        \left|e^{x+iy}\right| = \left|e^{x}e^{iy}\right|
        =
        \left|e^{x}\right|\left|\cos{y}+i\sin{y}\right|
        =
        e^{x} + \sqrt{ \cos^2{y} + \sin^2{y}} = e^{x}.
        \quad\square
        $$
    
\newpage

    \item[(e)] Prove that $e^{z}=1$ if and only if $z=2\pi ki$ for some $\in\mathbb{Z}$.

    \item[(f)] Show that every complex number $z=x+iy$ can be written in the form

        $$
        z=re^{i\theta}
        $$

    \item[(g)] In particular, $i=e^{i\pi/2}$. What is the geometric meaning of multiplying a
        complex number by $i$? Or by $e^{i\theta}$ for any $\theta\in\Real$?

    \item[(i)] Use the complex exponential to derive trigonometric identies such as
        $$
        \cos(\theta+\vartheta) = \cos\theta\cos\vartheta - \sin\theta\sin\vartheta
        $$

        and then show that

        $$
        2\sin\theta\sin\varphi = \cos(\theta-\varphi)-\cos(\theta+\varphi)
        $$
        $$
        2\sin\theta\cos\varphi = \sin(\theta+\varphi) + \sin(\theta-\varphi)
        $$

\end{enumerate}

\subsection{Chapter 1:5}

\subsection{Chapter 1:7}

\subsection{Chapter 1:10}

\newpage
