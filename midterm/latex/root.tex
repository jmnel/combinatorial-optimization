% ------------------
% -- Assignment 1 --
% -- Math 4171 -----

\documentclass[11pt,oneside]{article}

%\usepackage{subfigure}
\usepackage{subcaption}
\usepackage{graphicx}
\usepackage{amsmath}
\usepackage{amsfonts}
\usepackage{hyperref}
\usepackage{adjustbox}
\usepackage{listings}
\usepackage{xcolor}
\usepackage{titlesec}
\usepackage{enumitem}
\usepackage{mathrsfs}
\usepackage[driver=pdftex]{geometry}
\usepackage{import}
%\usepackage{titleformat{\section
%        {\normalfont\normalzie\bfseries}{Helo.}{1em}{}


\definecolor{codegreen}{rgb}{0,0.6,0}
\definecolor{codegray}{rgb}{0.5,0.5,0.5}
\definecolor{codepurple}{rgb}{0.58,0,0.82}
\definecolor{backcolour}{rgb}{0.95,0.95,0.92}
 
\lstdefinestyle{mystyle}{
    backgroundcolor=\color{backcolour},   
    commentstyle=\color{codegreen},
    keywordstyle=\color{magenta},
    numberstyle=\tiny\color{codegray},
    stringstyle=\color{codepurple},
    basicstyle=\ttfamily\footnotesize,
    breakatwhitespace=false,         
    breaklines=true,                 
    captionpos=b,                    
    keepspaces=true,                 
    numbers=left,                    
    numbersep=5pt,                  
    showspaces=false,                
    showstringspaces=false,
    showtabs=false,                  
    tabsize=2
}

\newtheorem{theorem}{Theorem}
\newtheorem{definition}{Definition}
\newtheorem{proof}{Proof}
 
\lstset{style=mystyle}

%\usepackage[margin=0.5in]{geometry}
\usepackage{inputenc}

\newcommand{\Real}{\mathbb{R}}
\newcommand{\Int}{\mathbb{Z}}
\newcommand{\Nat}{\mathbb{N}}
\newcommand{\Complex}{\mathbb{C}}
\newcommand{\vect}[1]{\boldsymbol{#1}}

\renewcommand{\Re}[1]{\mathfrak{Re}\left\lbrace{#1}\right\rbrace}
\renewcommand{\Im}[1]{\mathfrak{Im}\left\lbrace{#1}\right\rbrace}

\title{{\bf MATH 3172 3.0\\ Combinatorial Optimization}\\\vspace{10pt} \large Midterm I     
    \author{Jacques Nel}
}

\begin{document}

\maketitle

\newpage

%\renewcommand{\thesection}{Question \arabic{section}}

\section{Hill climb}

\subsection{\texttt{grid1} and \texttt{grid2}}

$$
\mathtt{grid1}=
\begin{bmatrix}
    3 & 7 & 2 & 8 \\
    5 & 2 & 9 & 1 \\
    5 & 3 & 3 & 1 \\
\end{bmatrix},
\quad\quad\text{and}\quad\quad
\mathtt{grid2}=
\begin{bmatrix}
    0 & 0 & 0 & 1 & 1 \\
    0 & 0 & 2 & 8 & 10 \\
    0 & 2 & 4 & 8 & 16 \\
    1 & 4 & 8 & 16 & 32 \\
\end{bmatrix}.
$$

The global maximum of \texttt{grid1} and \texttt{grid2} was trivial to find using
a naive implementation of the hill-climb. Both adjacent and diagonal state
transitions were allowed to reduce the number of iterations.

\begin{table}[h]
    \centering
    \caption{Hill-climb for three given discrete functions}
\begin{tabular}{c|cclr|c}
    Function $f(x)$ & iterations & time $(\mu s)$ & $x^\star$ & $f(x^\star)$ & success \\
    \hline
    \texttt{grid1} & 3 & 112 & $(1,2)$ & 9 & yes \\
    \texttt{grid2} & 2 & 39 & $(3,4)$ & 32 & yes \\
    \texttt{grid3} & 8 & 115 & $(7,98)$ & -7.4 & no \\
\end{tabular}
\end{table}

\newpage

\subsection{\texttt{grid3} is problematic}

Observe in the table, in the previous section, the naive hill climbing algorithm fails
to find the global maximum of \texttt{grid3} around the point $x^\star=(1,1)$.

\begin{figure}[h!]
    \centering
    \caption{3D plot of $f_3(\vect{x})$}
    \def\svgwidth{0.6\textwidth}
    \import{../figures/}{figure1-1.pdf_tex}
\end{figure}

\begin{figure}[h!]
    \centering
    \caption{$(-f_3(\vect{x}))^{1/8}$ to emphasize narrow global maximum band}
    \def\svgwidth{0.5\textwidth}
    \import{../figures/}{figure1-2.pdf_tex}
\end{figure}

Especially when discretized, \texttt{grid3} has a ridge on which $\vect{x}^\star=(1,1)$
lies. Hill climbing tends to get stuck along the sides this ridge, causing the
algorithm to fail to find the global maximum. Aliasing, as a result of discretization,
also results in several small isolated maxima near this ridge in the vicinity of 
$\vect{x}^\star$.

\newpage

\section{Simulated annealing}

\subsection{grid3}

\subsubsection{Results}

\begin{figure}[h!]
    \caption{Simulated annealing with adaptive exponential cooling schedule}
    \centering
    \begin{subfigure}{0.5\textwidth}%
        \centering
        \subcaption{State trajectory}
        \def\svgwidth{\textwidth}
        \import{../figures/}{figure3-1.pdf_tex}
    \end{subfigure}%
    \begin{subfigure}{0.5\textwidth}%
        \centering
        \subcaption{Adaptive-AE cooling schedule}
        \def\svgwidth{\textwidth}
        \import{../figures/}{figure3-2.pdf_tex}
    \end{subfigure}%
\end{figure}

An adaptive additive exponential cooling schedule was used to find the global maximum 
of \texttt{grid3}. Due to the large number of steps, the white line denotes an
exponential moving average or EMA of the trajectory with $\gamma=0.001$, instead of
showing the whole state trajectory.

\begin{equation}
    T_k = T_n + (T_0 - T_n)
    \left(\frac{1}{1+e^{\frac{2\ln\left(T_0-T_n\right)}{n}
    \left(k-\frac{1}{2}n\right)}}\right)
\end{equation}

Furthermore, we multiply $T_k$ by an adaptive term $1 \leq \mu \leq 2$ which
is calculated using the distance between the value of the current state $f(s_i)$ and
the best value encountered so far $f^\star$.

\begin{equation}
    T = \mu T_k =
    \left( 1 + \frac{f(s_i)-f^\star}{f(s_i)}\right)T_k
\end{equation}

In practice, we take the \texttt{np.abs} and use \texttt{np.clip(x, 1, 2)}
to ensure that these assumptions are maintained.

\subsubsection{Other cooling schedules considered}

 The following monotonic additive and multiplicative
cooling schedules were also tried, but failed to produce good results:

\begin{enumerate}
    \item Linear cooling,
    \item 
    \begin{enumerate}
        \item exponential multiplicative cooling,
        \item logarithmic multiplicative cooling,
        \item quadratic multiplicative cooling,
    \end{enumerate}
    \item
        \begin{enumerate}
            \item linear additive cooling,
            \item quadratic aditive cooling,
            \item exponential additive cooling.
        \end{enumerate}
\end{enumerate}

\end{document}
