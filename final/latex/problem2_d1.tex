\chapter{D-1 Wagon load balancing}

Three railyay wagons with a carrying capacity of 100 quintals (1 quintal = 100 kg)
have been reserved to transport sixteen boxes. The weight of the boxes in quintals
is given in the following table. How shall the boxes be assigned to the wagons
in order to keep to the limits on the maximum carrying capacity and to
minimize the heaviest wagon load?

\section{Parameters}

\begin{syms}

\item[$B$] enumerates $m=16$ boxes; ie. $B=\left\lbrace 1,\ldots, m\right\rbrace$

\item[$W$] enumerates $n=3$ railway wagons; ie. $W=\left\lbrace 1,\ldots, n\right\rbrace$

\item[$c$] denotes maximum capacity of a wagon in quintals, with $c=100$

\item[$\mu_b$] denotes the weight of box $b\in B$

\end{syms}

Refer to \cref{table:2-1} for the values of the above parameters.

\begin{table}[h]
    \center
    \caption{Weight of boxes}\label{table:2-1}
    \begin{tabu}{lcccccccc}
        \hline
        \rowfont[lcccccccc]{\bfseries} Box & 1 & 2 & 3 & 4 & 5 & 6 & 7 & 8 \\
        \textbf{Weight} & 34 & 6 & 8 & 17 & 16 & 5 & 13 & 21 \\
        \hline
        \rowfont[lcccccccc]{\bfseries} Box & 9 & 10 & 11 & 12 & 13 & 14 & 15 & 16 \\
        \textbf{Weight} & 25 & 31 & 14 & 13 & 33 & 9 & 25 & 25 \\
        \hline

    \end{tabu}
\end{table}

\section{Decision variable}

\begin{syms}

\item[$x_{bw}$] indicates box $b\in B$ is loaded in wagon $w\in W$; 
    $x_{bw}\in \left\lbrace 0, 1\right\rbrace$. For convience let
    $\vect{x} \in \left\lbrace 0, 1\right\rbrace^{m\times n}$
    with entries $x_{bw}, \forall (b,w) \in B\times W$.

\item[$m$] ancillary variable denoting upper bound of weights in
    quintals of all wagons;
    $0\leq m\in\Int$
\end{syms}

\section{Model}

\begin{mini!}
    {\vect{x}}{f(m)=m \protect\label{eq:d1-obj}}{\label{eq:d1}}{}
    \addConstraint{\sum_{w\in W}x_{bw}}{= 1, \forall b \in B \protect\label{eq:d1-ctr1}}
    \addConstraint{m}{\geq \sum_{b\in B}\mu_b x_{bw}, \forall w \in W \protect\label{eq:d1-ctr2}}
    \addConstraint{\sum_{b\in B}\mu_b x_{bw}}{\leq c, \forall w\in W\protect\label{eq:d1-ctr3}}
    \addConstraint{x_{bw}}{\in\left\lbrace 0, 1\right\rbrace, \forall \left(b,w\right)\in B\times W\protect\label{eq:d1-ctr4}}
    \addConstraint{m}{\geq 0 \protect\label{eq:d1-ctr5}}
\end{mini!}

The objective we week to minimize \cref{eq:d1-obj} is simply $m$ or rather,
the upper bound on the weights of each wagon $w\in W$. Each box $b\in B$ must
be loaded in exactly one wagon $w\in W$ as stated by \cref{eq:d1-ctr1}. 

For
$m$ to be the least upper bound of the weights of each wagon $w\in W$,
$m$ must be greater than all wagon weights as expressed in \cref{eq:d1-ctr2}. Minimization will lower it until
it is equal to the maximum of the weights of the loaded wagons. Note that this
objective is wrong in the text.

\Cref{eq:d1-ctr4} enforces that the decision variable $\vect{x}$ is a binary indicator,
and \cref{eq:d1-ctr5} is the canonical non-negativity constraint on $m$.

\section{Results}
