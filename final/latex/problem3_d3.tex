\chapter{D-3 Tank loading}

Five tanker ships have arrived at a chemical factory. They are carrying loads
of liquid products that must not be mixed. Refer to \cref{table:arrive} for
these quantities. Nine tanks of different capacities are available on site.
Some of them are already partially filled with a liquid. \Cref{table:tanks}
lists the characteristics of the tanks (in tonnes). Into which tanks should
the ships be unloaded (question 1) to maximize the capacity of the tanks
that remain unused, or (question 2) to maximize the number of tanks that remain
free?

\section{Question 1}

\subsection{Parameters}

\begin{syms}

\item[$T$] enumerates the $n=9$ tanks, ie. $T=\left\lbrace 1,\ldots,n\right\rbrace$

\item[$L$] enumerates the $m=5$ types of liquid arriving at the facility:
    $\left\lbrace \texttt{Benzol},\texttt{Butanol},\texttt{Propanol},
    \texttt{Styrene}, \texttt{THF}\right\rbrace$ ie.
    $L=\left\lbrace 1,\ldots, m\right\rbrace$.

\item[$F$] the indices of the prefilled tanks, ie. 
    $\left\lbrace \texttt{Benzol}, \texttt{THF}\right\rbrace$
    with $F = \left\lbrace T\in L : \text{tank } t \text{ is prefilled}\right\rbrace \subset T$

\item[$a_l$] denotes the amount (in tonnes) of liquid $l\in L$ arriving at the
    facility

\item[$c_t$] denotes the capacity (in tonnes) of tank $t\in T$

\item[$Q_t$] denotes the innitial amount (in tonnes) of liquid in tank $t\in F$

\item[$P_t$] denotes the type of liquid initially in tank $t\in F$

\item[$R_l$] remainder of liquid $l\in L$ after filling prefilled tanks $t\in F$ 
    to capacity with liquid of same type, given by


\begin{equation}
    R_l = \begin{cases}\displaystyle
        a_l - \sum_{t\in F\text, P_t = l} c_t - Q_t & \text{if }t\in F\\
        a_l & \text{otherwise}
    \end{cases}
\end{equation}

\end{syms}
    
Refer to \cref{table:arrive} for the values of $a_l$ and \cref{table:tanks} for
the values of $F, c_t, Q_t$, and $P_t$.

\begin{table}
    \center
    \caption{Various types of liquid arriving at facility}\label{table:arrive}
    \begin{tabu}{lcccccc}
        \hline
        \rowfont[lcccccc]{\bfseries} Type & Benzol & Butanol & Propanol & Styrene & THF \\
        \hline
        \textbf{Qauntity (in tonnes)} $a_l$ & 1200 & 700 & 1000 & 450 & 1200 \\
        \hline
    \end{tabu}
\end{table}

\begin{table}
    \center
    \caption{Facillity tank capacities and innitial quantities}\label{table:tanks}
    \begin{tabu}{lccccccccc}
        \hline
        \rowfont[lccccccccc]{\bfseries} Tank & 1 & 2 & 3 & 4 & 5 & 6 & 7 & 8 & 9\\
        \hline
        \textbf{Capacity (tonnes)} $c_t$ & 500 & 400 & 400 & 600 & 600 & 900 & 800 & 800 & 800 \\
        \textbf{Initial type} $Q_t$ & - & \texttt{Benzol} & - & - & - & - & \texttt{THF} & - & - \\
        \textbf{Initial amount (tonnes)} $P_t$ & 0 & 100 & 0 & 0 & 0 & 0 & 300 & 0 & 0 \\
        \hline
    \end{tabu}
\end{table}


\subsection{Decision variable}

\begin{syms}

\item[$x_{lt}$] indicates that remaining liquid of type $l\in L$ is pumped into
    tank $t\in T$, after prefilling; for convience let 
    $\vect{x}\in\left\lbrace 0, 1\right\rbrace^{m\times n}$ with entries $x_{lt}$

\end{syms}
\subsection{Model} \label{section:3.1.3}

\begin{mini!}
    {\vect{x}}{f(\vect{x})=\sum_{l\in L}\sum_{t\in T\setminus F} c_t x_{lt} \protect\label{eq:d3-q1-obj}}{\label{eq:d3-q1}}{}
    \addConstraint{\sum_{t\in T\setminus F} c_t x_{lt}}{& \geq R_l, \forall l\in L \protect\label{eq:d3-q1-ctr1}}
    \addConstraint{\sum_{l\in L}x_{lt}}{& \leq 1, \forall t\in T\setminus F \protect\label{eq:d3-q1-ctr2}}
    \addConstraint{x_{lt}}{& \in \left\lbrace 0, 1\right\rbrace, \forall (l,t)\in L\times T \protect\label{eq:d3-q1-ctr3}}
\end{mini!}

Maximizing the remaining capacity is equivalent to minmizing the used capacity
in a tank as expressed in \cref{eq:d3-q1-obj}. The family of constraints
\cref{eq:d3-q1-ctr1} ensures that all remaining liquid, after prefilling, is pumped
into tanks for all liquids $l\in L$. Liquids of different types can not be
allowed to mix in a tank. This is guaranteed by \cref{eq:d3-q1-ctr2}.
Lastly, \cref{eq:d3-q1-ctr3} states that $\vect{x}$ must be a binary variable.

\subsection{Results}

\section{Question 2}

We approach question 2 in almost an identical way as in question 1. The only thing
that changes is the objective function. The model for question 2 is given
below.

\subsection{Model}

\begin{mini!}
    {\vect{x}}{g(\vect{x})=\sum_{l\in L}\sum_{t\in T\setminus F} x_{lt} \protect\label{eq:d3-q2-obj}}{\label{eq:d3-q2}}{}
    \addConstraint{\sum_{t\in T\setminus F} c_t x_{lt}}{& \geq R_l, \forall l\in L \protect\label{eq:d3-q2-ctr1}}
    \addConstraint{\sum_{l\in L}x_{lt}}{& \leq 1, \forall t\in T\setminus F \protect\label{eq:d3-q2-ctr2}}
    \addConstraint{x_{lt}}{& \in \left\lbrace 0, 1\right\rbrace, \forall (l,t)\in L\times T \protect\label{eq:d3-q2-ctr3}}
\end{mini!}

\Cref{eq:d3-q2-obj} is similar to \cref{eq:d3-q1-obj} in \cref{section:3.1.3}.
The only difference between $g(\vect{x})$ and $f(\vect{x})$ is that $g\vect{x})$
does not include the tank capacities $c_t$. The result is, we are only counting
the number of tanks that used. Once again, minimizing this quantity is clearly
equivalent to maximizing the number of unused tanks.

\subsection{Results}
